\documentclass{article}
\usepackage{bm}
\usepackage{amsmath}
\usepackage{graphicx}
\usepackage{mdwlist}
\usepackage[colorlinks=true]{hyperref}
\usepackage{geometry}
\usepackage{kotex}
\geometry{margin=1in}
\geometry{headheight=2in}
\geometry{top=2in}
\usepackage{palatino}
%\renewcommand{\rmdefault}{palatino}
\usepackage{fancyhdr}
\usepackage{indentfirst}
\usepackage{multirow}
\usepackage{tabularx}

\newcommand{\red}[1]{{\color{red} #1}}
\newcommand{\blue}[1]{{\color{blue} #1}}
\newcommand{\orange}[1]{{\color{orange} #1}}
\newcommand{\purple}[1]{{\color{purple} #1}}

%\pagestyle{fancy}
\rhead{}
\lhead{}
\chead{%
  {\vbox{%
      \vspace{2mm}
      \large
      Statistics Lab 033.020\hfill
\\
      Seoul National University
      \\[4mm]
      \textbf{Assignment \#6} \\
      \texttt{2016-19516, Sangjun Son}
    }
  }
}

%%%%%%%%%%%%%%%%%%%%%%%
\usepackage{xcolor}
\usepackage{listings}
\definecolor{vgreen}{RGB}{104,180,104}
\definecolor{vblue}{RGB}{49,49,255}
\definecolor{vorange}{RGB}{255,143,102}

\lstdefinestyle{r-style}
{
    language=R,
    basicstyle=\footnotesize\ttfamily,
    keywordstyle=\color{vblue},
    identifierstyle=\color{black},
    commentstyle=\color{vgreen},
    numbers=left,
    numberstyle=\tiny\color{black},
    numbersep=10pt,
    tabsize=8,
    moredelim=*[s][\colorIndex]{[}{]},
    literate=*{:}{:}1,
    escapeinside=``,
}

\lstdefinestyle{out-style}
{
    language=R,
    basicstyle=\footnotesize\ttfamily,
    numbersep=10pt,
    tabsize=8,
    moredelim=*[s][\colorIndex]{[}{]},
    literate=*{:}{:}1,
    escapeinside=``,
}

\makeatletter
\newcommand*\@lbracket{[}
\newcommand*\@rbracket{]}
\newcommand*\@colon{:}
\newcommand*\colorIndex{%
    \edef\@temp{\the\lst@token}%
    \ifx\@temp\@lbracket \color{black}%
    \else\ifx\@temp\@rbracket \color{black}%
    \else\ifx\@temp\@colon \color{black}%
    \else \color{vorange}%
    \fi\fi\fi
}
\makeatother

\usepackage{trace}
%%%%%%%%%%%%%%%%%%%%%%%

\usepackage{paralist}

\usepackage{todonotes}
\setlength{\marginparwidth}{2.15cm}

\usepackage{tikz}
\usetikzlibrary{positioning,shapes,backgrounds}

\begin{document}

\pagestyle{fancy}

\section*{Example 1}
다음은 랜덤하게 선택된 806명의 Facebook 사용자들에 대해 Facebook의 privacy
setting 기능의 사용법을 알고 있는지에 대해 조사한 결과이다. 성별과 privacy setting의 사용법 인지 여부가 관계가 있다고 할 수 있는가? 유의수준 5\%에서 이를 검정하시오.
\begin{table}[htb!]
\centering
\begin{tabularx}{0.6\textwidth}{@{\extracolsep{\fill}}llccc}               &        & \multicolumn{2}{c}{성별} &     \\ \cline{3-4}
                                              &        & 남          & 여         & 합계  \\ \cline{2-5} 
\multicolumn{1}{c}{\multirow{3}{*}{사용법 인지여부}} & 알고 있다  & 288        & 378       & 666 \\
\multicolumn{1}{c}{}                          & 잘 모른다  & 10         & 7         & 17  \\
\multicolumn{1}{c}{}                          & 전혀 모른다 & 61         & 62        & 123 \\ \cline{2-5} 
                                              & 합계     & 359        & 447       & 806
\end{tabularx}
\end{table}

\begin{lstlisting}[style={r-style}]
facebook <- rbind(c(288, 278), c(10, 7), c(61, 62))
chisq.test(facebook)
\end{lstlisting}
\begin{lstlisting}[style={out-style}]
--------------------------------------------
	Pearson's Chi-squared test

data:  facebook
X-squared = 0.5104, df = 2, p-value = 0.7748
--------------------------------------------
\end{lstlisting}
\emph{Explanation: \textbf{독립성 검정 모형}, 남자가 Facebook의 privacy setting에 대해 알고 있을 확률, 잘 모를 확률, 전혀 모를 확률을 각각 $p_{11}, p_{21}, p_{31}$라 하고, 여자에 대해서도 마찬가지로 $p_{12}, p_{22}, p_{32}$ 라고 하면 검정하고자 하는 가설을 다음과 같다. 귀무가설 $H_0:$ 성별과 privacy setting의 사용법 인지 여부는 서로 독립이다와  대립가설 $H_1: \neg H_0$. 독립성 검정 chisq.test()를 진행하였을 때 유의확률은 0.7748로 유의수준 5\%에서 기각할 수 없다. 즉, 성별과 privacy setting의 사용법 인지 여부는 서로 독립이다.} \\

\section*{Example 2}
심폐 소생술은 심장과 폐의 활동이 갑자기 멈추었을 때 실시하는 응급처치방법이다.
심폐 소생술 시행 시 혈액 뭉침 등을 완화하기 위해 혈액 희석제(blood thinner)을 처방하는
경우가 있는데, 만약 심폐 소생술로 인해 신체 내부 손상을 입은 경우라면 혈액 희석제는 환
자에게 치명적일 수도 있다. 다음은 심폐 소생술을 받은 환자들 중 혈액 희석제를 사용하지
않은 환자들 50명과 혈액 희석제를 사용한 환자들 40명을 대상으로 그들의 생존 여부를 조사
한 결과이다.
\begin{table}[htb!]
\centering
\begin{tabularx}{0.6\textwidth}{@{\extracolsep{\fill}}lccc}
             & 생존 & 사망 & 합계 \\ \hline
혈액 희석제 사용 안함 & 11 & 39 & 50 \\
혈액 희석제 사용함   & 14 & 26 & 40 \\ \hline
합계           & 25 & 65 & 90
\end{tabularx}
\end{table}
혈액 희석제 사용 여부에 따른 생존률은 차이가 있는가? 유의수준 5\%에서 이를 검정하시오
\begin{lstlisting}[style={r-style}]
blood.thinner <- rbind(c(11, 39), c(14, 26))
chisq.test(blood.thinner)
\end{lstlisting}
\begin{lstlisting}[style={out-style}]
-----------------------------------------------------
	Pearson's Chi-squared test with Yates' continuity
	correction

data:  blood.thinner
X-squared = 1.2801, df = 1, p-value = 0.2579
-----------------------------------------------------
\end{lstlisting}
\emph{Explanation: \textbf{동질성 검정 모형}, 혈액 희석제를 사용하지 않은 그룹과 사용한 그룹을 먼저 50명과 40명으로 지정하였으므로 각 집단을 부차모집단으로 보고 혈액 희석제 사용 여부에 따른 생존률은 차이를 비교하면 된다. 이때 귀무가설 $H_0: p_{1j} = p_{2j} (j=1,2)$과 대립가설 $H_1: \neg H_0$로 설정 후 동질성 검정 chisq.test()를 진행한다. 자유도는 $df=(r-1)(c-1)=(2-1)(2-1)$} 임을 확인하고 유의확률은 0.2579로 유의수준 5\%에서 기각할 수 없다. 즉, 혈액 희석제 사용 여부에 따른 생존률은 차이가 존재한다고 할 수 없다. \\

\section*{Example 3}
3장 과제에서 사용하였던 iris 데이터셋을 분석하려고 한다. \texttt{data('iris')}의 코드를 실행하여 데이
터셋을 불러올 수 있으며, R에 내장된 iris 데이터에는 붓꽃의 종별 꽃잎과 꽃받침의 크기가 저장되어 있다. 이 중 \texttt{Sepal.Width}값이 3.3보다 크거나 같을 경우를 `L', 2.8보다 크거나 같거나
3.3보다 작은 경우를 `M', 2.8보다 작을 경우를 `S'로 분류할 때, setosa와 versicolor에 따른
붓꽃의 크기의 차이가 있다고 할 수있는지 유의수준 5\%에서 검정하여 보자. 
\begin{lstlisting}[style={r-style}]
data(iris)
setosa = iris[iris$Species == 'setosa',]
versicolor = iris[iris$Species == 'versicolor',]

setosa.L = setosa[setosa$Sepal.Width > 3.3,]
setosa.M = setosa[setosa$Sepal.Width >= 2.8 & setosa$Sepal.Width < 3.3,]
setosa.S = setosa[setosa$Sepal.Width < 2.8,]

versicolor.L = versicolor[versicolor$Sepal.Width > 3.3,]
versicolor.M = versicolor[versicolor$Sepal.Width >= 2.8 & versicolor$Sepal.Width < 3.3,]
versicolor.S = versicolor[versicolor$Sepal.Width < 2.8,]

iris.Sepal.Width = rbind(c(nrow(setosa.L), nrow(versicolor.L)),
                         c(nrow(setosa.M), nrow(versicolor.M)),
                         c(nrow(setosa.S), nrow(versicolor.S))); iris.Sepal.Width

chisq.test(iris.Sepal.Width)
\end{lstlisting}
\begin{lstlisting}[style={out-style}]
-----------------------------------------------
	Pearson's Chi-squared test

data:  iris.Sepal.Width
X-squared = 49.116, df = 2, p-value = 2.161e-11
-----------------------------------------------
\end{lstlisting}
\emph{Explanation: \textbf{동질성 검정 모형}, 예제 3과 동일한 방법으로 setosa의 집단과 versicolor 집단 내에서의 Sepal의 폭에 따라 크기를 나누었을 때의 그 비율이 동일한지에 대해 검정을 진행할 수 있다. `L', `M', `S'에 대한 조건 대로 setosa와 versicolor 집단에 대해 slicing을 진행하고 각각의 개체의 개수를 저장하였다. 이렇게 구한 값을 바탕으로 분할표를 rbind()와 c()를 통해 iris.Sepal.Width에 저장하였다. 동질성 검정 chisq.test()를 진행하였고 유의확률은 2.16e-11로 유의수준 5\%에서 기각할 수 있다. 즉, setosa와 versicolor에 따른 붓꽃의 크기의 차이가 존재한다. } \\

\section*{Example 4}
\texttt{vcd} 패키지 안에 있는 Arthritis 데이터 안에는 각 환자별 정보와 투약 여부에 따른 상태 호
전 결과가 저장되어 있다. 데이터의 남성환자에 대해서, 투약 여부(Treatment)에 따른 결과
(Improved)가 다르다고 할 수 있는지 유의수준 5\%에서 검정하여 보자. 
\begin{lstlisting}[style={r-style}]
install.packages("vcd")
library("vcd")
data(Arthritis)
male = Arthritis[Arthritis$Sex == "Male",]
Arthritis.table = table(male$Treatment, male$Improved); Arthritis.table
chisq.test(Arthritis.table)
\end{lstlisting}
\begin{lstlisting}[style={out-style}]
          None Some Marked
  Placebo   10    0      1
  Treated    7    2      5
----------------------------------------------
	Pearson's Chi-squared test

data:  Arthritis.table
X-squared = 4.9067, df = 2, p-value = 0.086
----------------------------------------------
\end{lstlisting}
\emph{Explanation: \textbf{동질성 검정 모형}, 예제 3, 4과 동일한 방법으로 남성 중 투약 한 집단 Treated과 그렇지 않은 집단 Placebo 내에서의 Improved 결과에 대한 비율이 동일한지에 대해 검정을 진행할 수 있다. 먼저 남성 데이터를 추출하기 위해 Arthritis의 Sex 변수를 통해 slicing을 진행하고 table()을 이용해 분할표 생성을 하였다. chisq() 검정 결과 유의확률은 0.086으로 유의수준 5\%에서 기각할 수 없다. 즉, 남성환자에 대해서, 투약 여부(Treatment)에 따른 결과
(Improved)가 다르다고 할 수 없다.} \\

\end{document}
