\documentclass{article}
\usepackage{bm}
\usepackage{amsmath}
\usepackage{graphicx}
\usepackage{mdwlist}
\usepackage[colorlinks=true]{hyperref}
\usepackage{geometry}
\usepackage{kotex}
\geometry{margin=1in}
\geometry{headheight=2in}
\geometry{top=2in}
\usepackage{palatino}
%\renewcommand{\rmdefault}{palatino}
\usepackage{fancyhdr}
\usepackage{indentfirst}

\newcommand{\red}[1]{{\color{red} #1}}
\newcommand{\blue}[1]{{\color{blue} #1}}
\newcommand{\orange}[1]{{\color{orange} #1}}
\newcommand{\purple}[1]{{\color{purple} #1}}

%\pagestyle{fancy}
\rhead{}
\lhead{}
\chead{%
  {\vbox{%
      \vspace{2mm}
      \large
      Statistics Lab 033.020\hfill
\\
      Seoul National University
      \\[4mm]
      \textbf{Assignment \#1} \\
      \texttt{2016-19516, Sangjun Son}
    }
  }
}

%%%%%%%%%%%%%%%%%%%%%%%
\usepackage{xcolor}
\usepackage{listings}
\definecolor{vgreen}{RGB}{104,180,104}
\definecolor{vblue}{RGB}{49,49,255}
\definecolor{vorange}{RGB}{255,143,102}

\lstdefinestyle{r-style}
{
    language=R,
    basicstyle=\scriptsize\ttfamily,
    keywordstyle=\color{vblue},
    identifierstyle=\color{black},
    commentstyle=\color{vgreen},
    numbers=left,
    numberstyle=\tiny\color{black},
    numbersep=10pt,
    tabsize=8,
    moredelim=*[s][\colorIndex]{[}{]},
    literate=*{:}{:}1
}

\lstdefinestyle{out-style}
{
    language=R,
    basicstyle=\scriptsize\ttfamily,
    numbersep=10pt,
    tabsize=8,
    moredelim=*[s][\colorIndex]{[}{]},
    literate=*{:}{:}1
}

\makeatletter
\newcommand*\@lbracket{[}
\newcommand*\@rbracket{]}
\newcommand*\@colon{:}
\newcommand*\colorIndex{%
    \edef\@temp{\the\lst@token}%
    \ifx\@temp\@lbracket \color{black}%
    \else\ifx\@temp\@rbracket \color{black}%
    \else\ifx\@temp\@colon \color{black}%
    \else \color{vorange}%
    \fi\fi\fi
}
\makeatother

\usepackage{trace}
%%%%%%%%%%%%%%%%%%%%%%%

\usepackage{paralist}

\usepackage{todonotes}
\setlength{\marginparwidth}{2.15cm}

\usepackage{tikz}
\usetikzlibrary{positioning,shapes,backgrounds}

\begin{document}

\pagestyle{fancy}

\section*{Example 1}
1부터 100 사이에서 40개의 정수를 임의로 선택하여 저장하자. 
\begin{lstlisting}[style={r-style}]
v = sample(1:100, 40, replace=F)
\end{lstlisting}
\emph{Explanation: 임의의 정수를 선택하기 위해 sample() 함수를 통해 정수배열을 생성하고 이를 v에 대입하였다.} \\

(\textbf{1-1}) 예제 1에서 만들어진 벡터를 사용하여 5행 8열의 행렬을 생성하시오.
\begin{lstlisting}[style={r-style}]
m = matrix(v, nrow=5, ncol=8)
m
\end{lstlisting}
\begin{lstlisting}[style={out-style}]
     [,1] [,2] [,3] [,4] [,5] [,6] [,7] [,8]
[1,]    6   74   82   34   17   40   88   36
[2,]   20    2   47    9   55   29   37   16
[3,]   59   81   28   77   85   80   71   89
[4,]   67   24   97    4   32   83   15    8
[5,]   86   10   93   63   13   53   76   14
\end{lstlisting}
\emph{Explanation: 40개의 정수 배열을 함수로 재생성하기 위해 matrix() 생성자로 $5\times8$ 행렬을 생성 후 m에 저장하고 출력하였다.} \\

(\textbf{1-2}) 
예제 1에서 생성된 행렬에서 2행과 3행만을 추출하여 저장하시오.
\begin{lstlisting}[style={r-style}]
m1 = m[2:3,]
m1
\end{lstlisting}
\begin{lstlisting}[style={out-style}]
     [,1] [,2] [,3] [,4] [,5] [,6] [,7] [,8]
[1,]   20    2   47    9   55   29   37   16
[2,]   59   81   28   77   85   80   71   89
\end{lstlisting}
\emph{Explanation: 위 예제에서 저장한 행렬 m의 $2 \sim 3$행을 가져오기 위해 indexing을 하여 m1에 저장 후 출력하였다.} \\

(\textbf{1-3}) 예제 1에서 생성된 행렬에서 (1, 4, 7, 8)열을 추출하여 저장하시오.
\begin{lstlisting}[style={r-style}]
m2 = m[,c(1,4,7,8)]
m2
\end{lstlisting}
\begin{lstlisting}[style={out-style}]
     [,1] [,2] [,3] [,4]
[1,]    6   34   88   36
[2,]   20    9   37   16
[3,]   59   77   71   89
[4,]   67    4   15    8
[5,]   86   63   76   14
\end{lstlisting}
\emph{Explanation: 행렬 m의 1, 4, 7, 8 열을 가져오기 위해 두 번째 인덱스에 대한 indexing을 하여 m2에 저장 후 출력하였다.} \\

(\textbf{1-4}) 예제 1에서 생성된 행렬의 7번째 열의 평균과 분산을 구하시오.
\begin{lstlisting}[style={r-style}]
mean(m[,7]); var(m[,7])
\end{lstlisting}
\begin{lstlisting}[style={out-style}]
[1] 57.4
[1] 920.3
\end{lstlisting}
\emph{Explanation: 행렬 m의 7열을 가져오기 위해 두 번째 인덱스에 대한 indexing을 하고 이렇게 생성된 벡터에 대한 평균과 분산을 구하기 위해 각각 mean(), var() 함수를 사용하여 구하였다.} \\

\section*{Example 2}
(\textbf{2-1}) 1부터 20 사이에서 1개의 숫자를 임의로 선택하여 저장하자. 이 숫자가 10 이상이면
“P”를 출력하고 10 미만이면 “NP”를 출력하는 코드를 작성해보시오.
\begin{lstlisting}[style={r-style}]
n = sample(1:20, 1, replace=F)
if (n >= 10) print("P") else print("NP")
\end{lstlisting}
\begin{lstlisting}[style={out-style}]
[1] "P"
\end{lstlisting}
\emph{Explanation: $1\sim20$ 사이의 숫자 1개를 선택하여 정수 n에 저장하고 if-else 문을 사용하여 n의 값이 10 이상일 경우 "P" 아니면 "NP"를 출력하였다.} \\

\section*{Example 3}
(\textbf{3-1}) 1부터 20 사이에서 8개의 숫자를 임의로 선택하여 저장하자. 8개의 숫자들 각각에
대해 숫자가 10 이상이면 “P”를 출력하고 10 미만이면 “NP”를 출력하는 코드를 작성해보시오.
\begin{lstlisting}[style={r-style}]
v = sample(1:20, 8, replace=F)
for (n in v) { if (n >= 10) print("P") else print("NP") }
\end{lstlisting}
\begin{lstlisting}[style={out-style}]
[1] "NP"
[1] "NP"
[1] "P"
[1] "P"
[1] "NP"
[1] "P"
[1] "P"
[1] "P"
\end{lstlisting}
\emph{Explanation: 위의 코드를 활용하면 조금 더 수월하게 8개 숫자에 대한 구현한 논리를 재사용할 수 있다. 한 숫자가 아닌 벡터에 대해 for 반복문을 통해 8개에 대한 "P" 또는 "NP"를 출력한다.} \\

\section*{Example 4}

(\textbf{4-1}) 1부터 100까지의 자연수를 n으로 저장하고, 그 합을 구하시오.
\begin{lstlisting}[style={r-style}]
n = 1:100
sum(n)
\end{lstlisting}
\begin{lstlisting}[style={out-style}]
[1] 5050
\end{lstlisting}
\emph{Explanation: $1\sim100$에 대한 모든 숫자를 벡터 n에 저장하고 벡터의 합을 sum()로 구하여 출력하였다.} \\

(\textbf{4-2}) 1부터 200까지의 자연수 중 3의 배수를 three로 저장하고, 그 합을 구하시오.
\begin{lstlisting}[style={r-style}]
three = seq(from=3, to=200, by=3)
sum(three)
\end{lstlisting}
\begin{lstlisting}[style={out-style}]
[1] 6633
\end{lstlisting}
\emph{Explanation: $1\sim200$ 사이의 모든 3의 배수를 저장하기 위해 seq()를 통해 초기값 3, 경계값 200, 간격 3을 주어 구간에 해당하는 모든 숫자 벡터를 three에 저장한다. three 벡터 원소의 합을 sum()로 구하여 출력하였다.} \\

\newpage
(\textbf{4-3}) 위 예제 4-2에서 저장한 three 벡터의 표준편차를 구하시오. 
단, 소수점 둘째자리까지 반올림한 값을 출력한다.
\begin{lstlisting}[style={r-style}]
round(sd(three), 2)
\end{lstlisting}
\begin{lstlisting}[style={out-style}]
[1] 57.59
\end{lstlisting}
\emph{Explanation: 벡터의 표준편차를 구하기 위해 sd() 함수를 사용하고 소수 둘째자리까지 반올림하여 표현한 실수값을 구하여 출력하였다.} \\

\section*{Example 5}
다음과 같은 행렬을 변수명 m으로 저장한다.
\begin{equation*}
    m = \begin{pmatrix}
        7 & 3 \\
        6 & 1 \\
        9 & 10
        \end{pmatrix}
\end{equation*}

(\textbf{5-1}) 행렬 m의 각 성분을 4로 나눈 나머지를 성분으로 갖는 행렬을 구하시오.
\begin{lstlisting}[style={r-style}]
m = cbind(c(7, 6, 9), c(3, 1, 10))
m %% 4
\end{lstlisting}
\begin{lstlisting}[style={out-style}]
     [,1] [,2]
[1,]    3    3
[2,]    2    1
[3,]    1    2
\end{lstlisting}
\emph{Explanation: 열 벡터를 만들어 열 방향으로 계속 쌓아 행렬을 만들기 위해 cbind()를 사용해 행렬 m을 생성하고 각 원소 별로 나머지 연산 \%\%을 수행하여 출력하였다.} \\

(\textbf{5-2}) 행렬 m의 1행에는 10씩을, 2행에는 20씩을, 3행에는 30씩을 더한 값을 성분으로 갖는 행렬을 구하시오.
\begin{lstlisting}[style={r-style}]
rbind(m[1,]+10, m[2,]+20, m[3,]+30)
\end{lstlisting}
\begin{lstlisting}[style={out-style}]
     [,1] [,2]
[1,]   17   13
[2,]   26   21
[3,]   39   40
\end{lstlisting}
\emph{Explanation: 각각의 행 방향 인덱싱을 통해 i행에 해당하는 벡터를 추출하고 원소별 덧셈 ($ + 10\cdot i$)을 하여 행 방향으로 쌓아 행렬을 만들고 출력하였다.} \\

\end{document}
